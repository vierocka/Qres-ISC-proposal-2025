% Options for packages loaded elsewhere
% Options for packages loaded elsewhere
\PassOptionsToPackage{unicode}{hyperref}
\PassOptionsToPackage{hyphens}{url}
\PassOptionsToPackage{dvipsnames,svgnames,x11names}{xcolor}
%
\documentclass[
]{article}
\usepackage{xcolor}
\usepackage[top=10pc,bottom=10pc,left=11pc,right=11pc,heightrounded,left=1.25in,right=1.25in]{geometry}
\usepackage{amsmath,amssymb}
\setcounter{secnumdepth}{-\maxdimen} % remove section numbering
\usepackage{iftex}
\ifPDFTeX
  \usepackage[T1]{fontenc}
  \usepackage[utf8]{inputenc}
  \usepackage{textcomp} % provide euro and other symbols
\else % if luatex or xetex
  \usepackage{unicode-math} % this also loads fontspec
  \defaultfontfeatures{Scale=MatchLowercase}
  \defaultfontfeatures[\rmfamily]{Ligatures=TeX,Scale=1}
\fi
\usepackage{lmodern}
\ifPDFTeX\else
  % xetex/luatex font selection
\fi
% Use upquote if available, for straight quotes in verbatim environments
\IfFileExists{upquote.sty}{\usepackage{upquote}}{}
\IfFileExists{microtype.sty}{% use microtype if available
  \usepackage[]{microtype}
  \UseMicrotypeSet[protrusion]{basicmath} % disable protrusion for tt fonts
}{}


\usepackage{longtable,booktabs,array}
\usepackage{calc} % for calculating minipage widths
% Correct order of tables after \paragraph or \subparagraph
\usepackage{etoolbox}
\makeatletter
\patchcmd\longtable{\par}{\if@noskipsec\mbox{}\fi\par}{}{}
\makeatother
% Allow footnotes in longtable head/foot
\IfFileExists{footnotehyper.sty}{\usepackage{footnotehyper}}{\usepackage{footnote}}
\makesavenoteenv{longtable}
\usepackage{graphicx}
\makeatletter
\newsavebox\pandoc@box
\newcommand*\pandocbounded[1]{% scales image to fit in text height/width
  \sbox\pandoc@box{#1}%
  \Gscale@div\@tempa{\textheight}{\dimexpr\ht\pandoc@box+\dp\pandoc@box\relax}%
  \Gscale@div\@tempb{\linewidth}{\wd\pandoc@box}%
  \ifdim\@tempb\p@<\@tempa\p@\let\@tempa\@tempb\fi% select the smaller of both
  \ifdim\@tempa\p@<\p@\scalebox{\@tempa}{\usebox\pandoc@box}%
  \else\usebox{\pandoc@box}%
  \fi%
}
% Set default figure placement to htbp
\def\fps@figure{htbp}
\makeatother


% definitions for citeproc citations
\NewDocumentCommand\citeproctext{}{}
\NewDocumentCommand\citeproc{mm}{%
  \begingroup\def\citeproctext{#2}\cite{#1}\endgroup}
\makeatletter
 % allow citations to break across lines
 \let\@cite@ofmt\@firstofone
 % avoid brackets around text for \cite:
 \def\@biblabel#1{}
 \def\@cite#1#2{{#1\if@tempswa , #2\fi}}
\makeatother
\newlength{\cslhangindent}
\setlength{\cslhangindent}{1.5em}
\newlength{\csllabelwidth}
\setlength{\csllabelwidth}{3em}
\newenvironment{CSLReferences}[2] % #1 hanging-indent, #2 entry-spacing
 {\begin{list}{}{%
  \setlength{\itemindent}{0pt}
  \setlength{\leftmargin}{0pt}
  \setlength{\parsep}{0pt}
  % turn on hanging indent if param 1 is 1
  \ifodd #1
   \setlength{\leftmargin}{\cslhangindent}
   \setlength{\itemindent}{-1\cslhangindent}
  \fi
  % set entry spacing
  \setlength{\itemsep}{#2\baselineskip}}}
 {\end{list}}
\usepackage{calc}
\newcommand{\CSLBlock}[1]{\hfill\break\parbox[t]{\linewidth}{\strut\ignorespaces#1\strut}}
\newcommand{\CSLLeftMargin}[1]{\parbox[t]{\csllabelwidth}{\strut#1\strut}}
\newcommand{\CSLRightInline}[1]{\parbox[t]{\linewidth - \csllabelwidth}{\strut#1\strut}}
\newcommand{\CSLIndent}[1]{\hspace{\cslhangindent}#1}



\setlength{\emergencystretch}{3em} % prevent overfull lines

\providecommand{\tightlist}{%
  \setlength{\itemsep}{0pt}\setlength{\parskip}{0pt}}





% -----------------------
% CUSTOM PREAMBLE STUFF
% -----------------------

% -----------------
% Typography tweaks
% -----------------
% Indent size
\setlength{\parindent}{1pc}  % 1p0

% Fix widows and orphans
\usepackage[all,defaultlines=2]{nowidow}

% List things
\usepackage{enumitem}
% Same document-level indentation for ordered and ordered lists
\setlist[1]{labelindent=\parindent}
\setlist[itemize]{leftmargin=*}
\setlist[enumerate]{leftmargin=*}

% Wrap definition list terms
% https://tex.stackexchange.com/a/9763/11851
\setlist[description]{style=unboxed}


% For better TOCs
\usepackage[titles]{tocloft}

% Remove left margin in lists inside longtables
% https://tex.stackexchange.com/a/378190/11851
\AtBeginEnvironment{longtable}{\setlist[itemize]{nosep, wide=0pt, leftmargin=*, before=\vspace*{-\baselineskip}, after=\vspace*{-\baselineskip}}}

% Allow for /singlespacing and /doublespacing
\usepackage{setspace}


% -----------------
% Title block stuff
% -----------------

% Abstract
\usepackage[overload]{textcase}
\usepackage[runin]{abstract}
\renewcommand{\abstractnamefont}{\sffamily\footnotesize\bfseries\MakeUppercase}
\renewcommand{\abstracttextfont}{\sffamily\small}
\setlength{\absleftindent}{\parindent * 2}
\setlength{\absrightindent}{\parindent * 2}
\abslabeldelim{\quad}
\setlength{\abstitleskip}{-\parindent}


% Keywords
\newenvironment{keywords}
{\vskip -3em \hspace{\parindent}\small\sffamily{\sffamily\footnotesize\bfseries\MakeUppercase{Keywords}}\quad}
{\vskip 3em}

  
% Title
\usepackage{titling}
\setlength{\droptitle}{3em}
\pretitle{\par\vskip 5em \begin{flushleft}\LARGE\sffamily\bfseries}
\posttitle{\par\end{flushleft}\vskip 0.75em}


% Authors
%
% PHEW this is complicated for a number of reasons!
%
% When using \and with multiple authors, the article class in LaTeX wraps each 
% author block in a tabluar environment with a hardcoded center alignment. It's 
% possible to use \preauthor{} to start tabulars with a left alignment {l}, but 
% that only applies to the first author because the others all use \and with the 
% hardcoded {c}. But we can override the \and command and add our own {l}
%
% (See https://github.com/rstudio/rmarkdown/issues/1716#issuecomment-560601691 
% for an example of redefining \and to just be \\)
%
% That's all great, except tabulars have some amount of default horizontal 
% padding, which makes left-aligned author blocks not actuall get fully 
% left-aligned on the page. We can set the horizontal padding for the column to 
% 0, but it requires some wonky syntax: {@{\hspace{0em}}l@{}}
\renewcommand{\and}{\end{tabular} \hskip 3em \begin{tabular}[t]{@{\hspace{0em}}l@{}}}
\preauthor{\begin{flushleft}
           \lineskip 1.5em 
           \begin{tabular}[t]{@{\hspace{0em}}l@{}}}
\postauthor{\end{tabular}\par\end{flushleft}}

% Omit the date since the \published command does that
\predate{}
\postdate{}

% Command for a note at the top of the first page describing the publication
% status of the paper.
\newcommand{\published}[1]{%
   \gdef\puB{#1}}
   \newcommand{\puB}{}
   \renewcommand{\maketitlehooka}{%
       \par\noindent\footnotesize\sffamily \puB}


% ------------------
% Section headings
% ------------------
\usepackage{titlesec}
\titleformat*{\section}{\Large\sffamily\bfseries\raggedright}
\titleformat*{\subsection}{\large\sffamily\bfseries\raggedright}
\titleformat*{\subsubsection}{\normalsize\sffamily\bfseries\raggedright}
\titleformat*{\paragraph}{\small\sffamily\bfseries\raggedright}

% \titlespacing{<command>}{<left>}{<before-sep>}{<after-sep>}
% Starred version removes indentation in following paragraph
\titlespacing*{\section}{0em}{2em}{0.1em}
\titlespacing*{\subsection}{0em}{1.25em}{0.1em}
\titlespacing*{\subsubsection}{0em}{0.75em}{0em}


% -----------
% Footnotes
% -----------
% NB: footmisc has to come after setspace and biblatex because of conflicts
\usepackage[bottom, flushmargin]{footmisc}
\renewcommand*{\footnotelayout}{\footnotesize}

\addtolength{\skip\footins}{10pt}    % vertical space between rule and main text
\setlength{\footnotesep}{5pt}  % vertical space between footnotes


% ----------
% Captions
% ----------
\usepackage[font={small,sf}, labelfont={small,sf,bf}]{caption}


% --------
% Macros
% --------
% pandoc will not convert text within \begin{} XXX \end{} to Markdown and will
% treat it as regular TeX. Because of this, it's impossible to do stuff like
% this:

% \begin{landscape}
%
% | One | Two   |
% |-----+-------|
% | my  | table |
% | is  | nice  |
%
% \end{landscape}
%
% Since it'll render like: | One | Two | |—–+——-| | my | table | | is | nice |
% 
% BUT, from this http://stackoverflow.com/a/41945462/120898 we can get around
% this by creating new commands for \begin and \end, like this:
\usepackage{pdflscape}
\newcommand{\blandscape}{\begin{landscape}}
\newcommand{\elandscape}{\end{landscape}}

% \blandscape
%
% | One | Two   |
% |-----+-------|
% | my  | table |
% | is  | nice  |
%
% \elandscape

% Same thing, but for generic groups
% But can't use \bgroup and \egroup because those are built-in TeX things
\newcommand{\stgroup}{\begingroup}
\newcommand{\fingroup}{\endgroup}


% ---------------------------
% END CUSTOM PREAMBLE STUFF
% ---------------------------
\makeatletter
\@ifpackageloaded{caption}{}{\usepackage{caption}}
\AtBeginDocument{%
\ifdefined\contentsname
  \renewcommand*\contentsname{Table of contents}
\else
  \newcommand\contentsname{Table of contents}
\fi
\ifdefined\listfigurename
  \renewcommand*\listfigurename{List of Figures}
\else
  \newcommand\listfigurename{List of Figures}
\fi
\ifdefined\listtablename
  \renewcommand*\listtablename{List of Tables}
\else
  \newcommand\listtablename{List of Tables}
\fi
\ifdefined\figurename
  \renewcommand*\figurename{Figure}
\else
  \newcommand\figurename{Figure}
\fi
\ifdefined\tablename
  \renewcommand*\tablename{Table}
\else
  \newcommand\tablename{Table}
\fi
}
\@ifpackageloaded{float}{}{\usepackage{float}}
\floatstyle{ruled}
\@ifundefined{c@chapter}{\newfloat{codelisting}{h}{lop}}{\newfloat{codelisting}{h}{lop}[chapter]}
\floatname{codelisting}{Listing}
\newcommand*\listoflistings{\listof{codelisting}{List of Listings}}
\makeatother
\makeatletter
\makeatother
\makeatletter
\@ifpackageloaded{caption}{}{\usepackage{caption}}
\@ifpackageloaded{subcaption}{}{\usepackage{subcaption}}
\makeatother
\usepackage{bookmark}
\IfFileExists{xurl.sty}{\usepackage{xurl}}{} % add URL line breaks if available
\urlstyle{same}
\hypersetup{
  pdftitle={QRes - An Open-Source Dose--Response Data Platform for the R Community},
  pdfauthor={Viera Kovacova},
  colorlinks=true,
  linkcolor={DarkSlateBlue},
  filecolor={Maroon},
  citecolor={DarkSlateBlue},
  urlcolor={DarkSlateBlue},
  pdfcreator={LaTeX via pandoc}}


% -----------------------
% END-OF-PREAMBLE STUFF
% -----------------------



% ---------------------- 
% Title block elements
% ---------------------- 
\usepackage{orcidlink}  % Create automatic ORCID icons/links

\title{QRes - An Open-Source Dose--Response Data Platform for the R
Community}


\author{
{\large Viera Kovacova~\orcidlink{0000-0003-4581-4254}}%
 \\%
Viera Kovacova, Institute for Biological Physics, University of Cologne,
Cologne, Germany \\%
{\footnotesize \url{}} \and
}

\date{}


% Typeset URLs in the same font as their parent environment
%
% This has to come at the end of the preamble, after any biblatex stuff because 
% some biblatex styles (like APA) define their own \urlstyle{}
\usepackage{url}
\urlstyle{same}

% ---------------------------
% END END-OF-PREAMBLE STUFF
% ---------------------------
\begin{document}
% ---------------
% TITLE SECTION
% ---------------
\published{\textbf{2025-09-29}}

\maketitle



% -------------------
% END TITLE SECTION
% -------------------



\section{Executive Summary}\label{executive-summary}

Dose--response data are fundamental to research in microbiology,
oncology, pharmacology, biophysics, and toxicology; yet, raw datasets
remain fragmented, inconsistently formatted, and often lack metadata,
making them challenging to integrate into reproducible R workflows. QRes
platform is a web-based platform hosted at the University of Cologne
that collects, processes, and analyzes dose--response datasets. This
proposal seeks R Consortium support to expand QRes into a fully
R-integrated, open-source resource for the global R community.

The project aims to deliver a public API and an R package designed for
standardized data access, database storage, and data analysis.
Additionally, it will provide an enhanced R Shiny interface that allows
users to both upload and download raw data along with its associated
metadata, as well as query and analyze this data. This interface will
include an automated submission form for validating and storing new
datasets, as well as tools for interactive visualization and analysis of
dose--response measurements, all accessible directly through the
browser.

QRes will integrate predictive tools developed in Python and Julia by
Prof.~Ville Mustonen (FiMAR, Helsinki) and Dr.~Fernanda Pinheiro (Human
Technopol, Milan), enabling cross-language interoperability. In
addition, the DGrowthR package (developed by Medina Feldl at Helmholtz,
Munich) will be incorporated as an analysis option for bulk
dose--response data. Key outputs include an R package released on CRAN,
standardized data formats, reproducible workflows, and community
guidelines establishing best practices and gold standards for
dose--response measurements. By bridging R with Python and Julia tools,
Qres will provide a foundation for AI-driven analysis and long-term
reproducible science. This project directly addresses the R Consortium's
mission by strengthening technical infrastructure, fostering
interoperability, and serving a broad international user base across
biology, biophysics, and medicine.

The Qres platform is an open-access dose--response database and web
service designed to transform how raw microbial and cancer cell growth
curve data are collected, curated, and analyzed. Measuring
dose--response curves is a fundamental experiment in microbiology,
oncology, biophysics, and pharmacology; yet, access to raw time-resolved
data is limited, often buried in supplementary files and lacking
standardized formats. This fragmentation hampers reproducibility,
inhibits cross-study comparisons, and slows progress in understanding
resistance dynamics and cellular responses under perturbation.

Qres addresses these challenges by providing a centralized, open, and
R-integrated platform that: * Collects raw, time-resolved growth curves
under antibiotic, chemical, genetic, or environmental perturbations. *
Offers an automated Shiny-based submission form for uploading and
validating raw measurements with curated metadata. * Implements
standardized pipelines for extracting quantitative resistance traits and
automating the fitting and benchmarking of growth curve models. *
Provides a public API and R package to programmatically query datasets,
returning either complete raw data or derived parameters. * Integrates
curve fitting and predictive modeling tools developed in Python and
Julia by Prof.~Ville Mustonen (FiMAR, Helsinki) and Dr.~Fernanda
Pinheiro (Human Technopol, Milan), reimplemented or wrapped for use in
R. * Connects with the DGrowthR package (Medina Feldl, Helmholtz,
Munich) for analysis of microbial growth data.

Expected outcomes and deliverables include: * An R package released on
CRAN with functions for standardized data access, preprocessing, and
model fitting. * A robust R Shiny interface enabling data
upload/download, metadata validation, and interactive visualization. * A
curated, FAIR-compliant database with persistent identifiers for
citation and cross-study reuse. * Benchmarked integration of
cross-language predictive tools within R workflows. * Published
best-practice guidelines and gold standards for microbial dose--response
experiments.

Budget and Team: Funding is requested to support two PhD-level staff
members for 12 months --- one focusing on database/software development,
and the other on R, Shiny, and API development. The PI (Viera Kováčová,
University of Cologne) will supervise the project, but does not request
salary support. Collaborations with FiMAR (Finland), Human Technopol
(Italy), and Helmholtz (Germany) provide complementary expertise in
predictive modeling and growth analysis.

By creating an open, standardized, and interactive infrastructure, Qres
will accelerate reproducible research, establish community standards,
and strengthen R's role as the primary platform for dose--response data
analysis, serving a broad international user base across microbiology,
oncology, pharmacology, and beyond.

\section{Signatories}\label{signatories}

This section summarizes the individuals and groups who support,
contribute to, or have been consulted on the Qres proposal. The project
is already embedded in large research networks (CRC1310, CRC1678, FiMAR,
Human Technopol, Helmholtz) and has broad community visibility.

\subsection{Project team}\label{project-team}

\begin{itemize}
\tightlist
\item
  Viera Kováčová, PhD (viera.kovacova@uni-koeln.de) -- Principal
  Investigator. Leads the development of Qres, supervises technical and
  scientific progress, and coordinates collaborations.
\item
  Prof.~Michael Lässig (lassig@uni-koeln.de) -- Director of CRC1310.
  Provides strategic oversight and advice on the broader scientific
  vision.
\item
  Stephan Kleinboltig (kleinboltig@uni-koeln.de) -- CRC1310 project
  manager. Supports administrative coordination and R server
  infrastructure at the University of Cologne.
\end{itemize}

Two PhD Students (to be hired) -- Dedicated to (1) software development
and database/API implementation, and (2) R/Shiny/API development and
user interface.

\subsection{Contributors}\label{contributors}

\begin{itemize}
\tightlist
\item
  Prof.~Ville Mustonen (FiMAR, Helsinki) -- Provides predictive modeling
  expertise, helps shape the scientific scope, and engages international
  collaborators to contribute datasets.
\item
  Dr.~Fernanda Pinheiro (Human Technopol, Milan) -- Leads development of
  the Kinbiont Julia package for bacterial growth modeling; collaborates
  on integration into Qres.
\item
  Prof.~Tobias Bollenbach (University of Cologne) -- CRC1310 PI
  producing large-scale dose--response datasets; supports data
  contribution and benchmarking.
\end{itemize}

\subsection{Consulted}\label{consulted}

University of Cologne colleagues, all contributing feedback and testing:
- Theresa Finke - Leon Sieger - Gabriela Pertungaro - Rotem Gross

\begin{itemize}
\tightlist
\item
  C3RDM and ITCC departments, University of Cologne -- Consulted on IT
  infrastructure, data management, and server support.
\item
  Medina Feldl (Helmholtz, Munich) -- Developer of the DGrowthR package;
  contributes to Qres integration for bacterial dose--response analysis.
\item
  Prof.~Andreas Beyer (andreas.beyer@uni-koeln.de) -- Director of
  CRC1678. Advises on systems biology integration and data-driven
  approaches.
\item
  Prof.~Miroslav Baránek (Mendel University, Lednice, Czech Republic) --
  Provides perspective on dose--response measurements in plant
  pathogens.
\end{itemize}

Industry contacts: representatives from Agilent, Tecan, Molecular
Devices, Promega, and BMGLabtech -- provided test/sample files from
microplate readers and laboratory robots to inform data ingestion
pipelines.

\subsection{Community Support}\label{community-support}

The project is known and endorsed within CRC1310 Area A (all PIs listed
at crc1310.uni-koeln.de/project\_a.html ). This broad base of support
ensures that Qres is well-positioned for adoption and sustainability,
with immediate access to diverse microbial dose--response datasets.

\section{The Problem}\label{the-problem}

Dose--response measurements are among the most fundamental experiments
in biology: they quantify how cells, microbes, and tissues respond to
drugs, antibiotics, toxins, or environmental perturbations. These data
are critical for understanding microbial resistance, cancer treatment
response, and pharmacological efficacy. Despite their central role, the
raw growth curve data underlying dose--response studies remain
fragmented, inconsistently reported, and difficult to reuse.

Who it affects:

\begin{enumerate}
\def\labelenumi{\arabic{enumi}.}
\tightlist
\item
  Experimental researchers face difficulties identifying optimal
  concentrations or conditions because raw data are buried in
  supplementary materials or inaccessible formats.
\item
  Modelers and computational biologists cannot easily benchmark or
  parameterize models due to lack of standardized, cross-study datasets.
\item
  The R community has strong modeling tools (e.g.~drc, nlme, brms) but
  lacks a central resource of curated datasets with reproducible access
  pathways.
\end{enumerate}

Why it is a problem:

\begin{enumerate}
\def\labelenumi{\arabic{enumi}.}
\tightlist
\item
  Lack of accessible raw data prevents reproducibility and cross-study
  comparison.
\item
  Metadata are often incomplete, making it impossible to assess
  experimental context.
\item
  Absence of standards leads to inconsistent extraction of resistance
  traits and dose--response parameters.
\end{enumerate}

What solving the problem enables:

\begin{enumerate}
\def\labelenumi{\arabic{enumi}.}
\tightlist
\item
  Establishing Qres will provide the R community with a curated,
  FAIR-compliant infrastructure for dose--response data.
\item
  Researchers will be able to upload and share raw datasets with
  metadata, ensuring proper citation and recognition.
\item
  Modelers will have standardized, comparable inputs for parameter
  estimation, benchmarking, and AI-driven predictions.
\item
  R users will benefit from integrated pipelines linking raw data to
  statistical and mechanistic modeling.
\item
  For society, Qres will deepen our understanding of microbial
  resistance and its evolution, support better-informed strategies to
  combat antibiotic resistance, and provide benchmarks that improve the
  design of therapeutic interventions. By enabling systematic
  cross-study comparisons, Qres will help identify resistance patterns
  earlier, guide clinical and experimental decisions, and ultimately
  contribute to public health.
\end{enumerate}

Existing work: 1. R packages like drc (\citeproc{ref-ritz2015}{Ritz et
al. 2015}), nlme (\citeproc{ref-pinheiro2023}{Pinheiro and Bates 2023}),
and brms (\citeproc{ref-buxfcrkner2017}{Bürkner 2017}) support model
fitting, but they assume users already have curated datasets. 2.
DGrowthR (Helmholtz) (\citeproc{ref-feldl2025}{Feldl and Olayo Alarcon
2025}) provides functions for analyzing microbial growth, but it does
not provide access to standardized raw data repositories.

To date, there is no centralized, open-access platform in R that
combines data storage, metadata standards, reproducible access, and
interactive visualization. This gap limits reproducibility, slows the
pace of discovery, and prevents the R community from fully supporting
the next generation of data-driven microbiology and pharmacology.

\section{The proposal}\label{the-proposal}

\subsection{Approach and Work Plan}\label{approach-and-work-plan}

We will address the problem of fragmented and inaccessible
dose--response datasets by developing Qres into an open, standardized,
R-integrated platform with the following concrete actions: ** API and
Database Development (Months 1--4) * Implement a REST API with
standardized JSON schema. * Expand the database schema to store raw
growth curves and curated metadata. * Establish FAIR-compliant pipelines
for data ingestion. ** R Package (Months 5--6) * Develop an open-source
R package (QresR) to query the database and API. * Provide functions for
data preprocessing, extraction of resistance traits, and integration
with existing modeling packages (drc, nlme, brms). * Begin integration
of DGrowthR for bulk growth curve analysis. ** Shiny Interface and
Submission Form (Months 7--8) * Extend the existing Shiny prototype into
a full interface supporting upload and download of raw growth data with
metadata. * Implement automated submission validation and error
reporting. * Add interactive visualization and curve-fitting modules. **
Cross-Language Integration (Months 9--10) * Reimplement or wrap selected
Python/Julia predictive tools developed by collaborators (Mustonen,
Pinheiro). * Benchmark results across platforms to ensure
reproducibility. ** Testing, Standards, and Dissemination (Months
11--12) * Pilot test data submissions with internal and external
collaborators. * Optimize submission form and analysis workflows. *
Publish the R package on CRAN, complete documentation, tutorials, and
case studies. * Release community guidelines establishing gold standards
for dose--response experiments.

\subsection{Risk Management}\label{risk-management}

1.) Risk: Data heterogeneity -\textgreater{} Mitigation: enforce a
standardized metadata schema and automated validation in the Shiny
submission form. 2.) Risk: Integration complexity of Python/Julia tools
-\textgreater{} Mitigation: begin with a small subset of methods and
provide R wrappers before attempting full reimplementations. 3.) Risk:
Limited adoption -\textgreater{} Mitigation: leverage CRC1310, FiMAR,
and Human Technopol networks to ensure early use and dataset
contributions; provide persistent identifiers for proper citation.

\subsection{Timeline}\label{timeline}

Months 1--4: Database + API development. Months 5--6: R package
prototype. Months 7--8: Shiny submission system. Months 9--10:
Cross-language integration. Month 11: Testing and optimization. Month
12: Release, dissemination, and publication of guidelines.

\subsection{Overview}\label{overview}

The Qres platform is an open-access dose--response database and web
service that collects, organizes, and analyzes raw, time-resolved
microbial growth curves under antibiotic, genetic, and environmental
perturbations. Currently, raw dose--response data are fragmented,
inconsistently reported, and often hidden in supplementary files, making
it difficult for researchers to reproduce analyses, perform cross-study
comparisons, or build robust models of microbial resistance and cancer
evolution.

Qres directly addresses this problem by providing the R community with a
standardized, open-source infrastructure for dose--response data. The
platform combines a robust database and public API, an R package (QresR)
for reproducible workflows, and an extended Shiny interface that enables
both upload and download of raw data with curated metadata, along with
interactive visualization and analysis.

The project is developed in collaboration with international partners:
Predictive tools from FiMAR (Helsinki) and Human Technopol (Milan) will
be reimplemented or wrapped for integration into R. The DGrowthR package
(Helmholtz, Munich) will be incorporated for bacterial growth curve
analysis.

Benefits to the R community: * Access to curated, FAIR-compliant raw
dose--response datasets. * Reproducible workflows for microbial growth,
cancel cell growth and resistance modeling. * Integration of
cross-language predictive tools into R. * Community guidelines
establishing gold standards for dose--response data.

By solving the accessibility and reproducibility gap in dose--response
research, the Qres will accelerate discovery, improve experimental
design, and expand R's role as the central platform for quantitative
biology (for example, in the field of antibiotic resistance).

\subsection{Detail}\label{detail}

\subsubsection{Minimum Viable Product}\label{minimum-viable-product}

The smallest version of Qres that delivers value to the R community will
include: 1. A database + REST API for storing and retrieving
standardized dose--response datasets. 2. An R package (QresR) providing
functions to query the API and return data in tidy formats for
downstream analysis with packages like drc, nlme, and brms. 3. A basic
Shiny interface that allows users to upload raw growth curve data with
metadata, validates submissions, and enables download of standardized
datasets.

This MVP ensures that from the beginning, R users can both contribute
new datasets and reuse existing data in reproducible workflows.

\subsubsection{Architecture}\label{architecture}

At a high level, the architecture will consist of: 1. Database Layer *
Stores raw growth measurements, curated metadata, and derived resistance
traits. * Implements FAIR-compliant schemas with persistent identifiers
for citation. 2. API Layer * RESTful API (JSON-based) for standardized
programmatic access. * Enables both raw data retrieval and derived
parameter queries. 3. R Package (QresR) * Provides user-facing functions
to query the API, clean and preprocess data, and connect to modeling
packages. * Interfaces with external analysis tools such as DGrowthR. 4.
Web Interface (Shiny) * Supports upload of raw data + metadata via
automated submission form. * Performs validation, error reporting, and
visualization of fitted curves. * Allows users to explore and download
datasets interactively. 5. External Tool Integration * Wrappers or
reimplementations for Python/Julia tools (Mustonen, Pinheiro). *
Optional DGrowthR workflows for bulk analysis.

\subsubsection{Assumptions}\label{assumptions}

The project assumes that: 1. Community willingness: Researchers will
contribute raw dose--response datasets when given easy submission tools
and proper citation credit (via persistent identifiers). 2. Technical
feasibility: Core predictive tools from Python/Julia can be either
wrapped or reimplemented in R without loss of fidelity. 3. Resource
stability: Hosting on University of Cologne servers remains available
for the duration of the project, with minimal cost. 4. Sufficient
adoption: By leveraging CRC1310, FiMAR, and Human Technopol networks,
Qres will gain early adopters who validate and stress-test the system.

If any of these assumptions were false (e.g.~very low data submission,
or inability to wrap external tools), Qres would still deliver value as
a curated database + API + R package, but the broader interoperability
and uptake could be delayed.

\subsubsection{External dependencies}\label{external-dependencies}

Qres builds on several external components: 1. R ecosystem: Packages
including drc, nlme, brms, and shiny for modeling and visualization. 2.
DGrowthR (Helmholtz, Munich): External R package for analysis,
integrated into Qres workflows. 3. Python/Julia predictive tools:
Developed by collaborators at FiMAR (Helsinki) and Human Technopol
(Milan); require wrapping or partial reimplementation in R. 4. Server
infrastructure: Hosted on University of Cologne's R Shiny server, with
additional database backend. 5. Community datasets: Contributions from
CRC1310, FiMAR, Human Technopol, and beyond are essential to seed the
database.

\section{Project plan}\label{project-plan}

\subsection{Start-up phase}\label{start-up-phase}

Collaboration platform: We will host all code on GitHub under an open
repository (qres-platform), including the API, Shiny interface, and R
package. Issues and pull requests will be used to manage contributions
from collaborators and the community.

License decisions: All code will be released under the MIT license; data
schemas and metadata standards will be released under CC-BY 4.0 to
maximize reuse.

Reporting framework: We will provide quarterly progress updates to the
ISC, including blog posts summarizing milestones. The GitHub repository
will include a project board to track deliverables.

\subsection{Technical delivery}\label{technical-delivery}

Months 1--2: Finalize schema design, API blueprint, GitHub repo setup,
and automated CI/CD pipeline for the R package.

Months 3--4: Implement database backend and API endpoints; first
internal prototype of data upload form.

Months 5--6: Release alpha version of QresR package (GitHub); API
documentation available online.

Months 7--8: Extend Shiny interface with validated data submission and
visualization modules; first external test submissions.

Months 9--10: Integrate predictive tools (Python/Julia) and DGrowthR;
provide R wrappers.

Months 11--12: Community testing with CRC1310/FiMAR/Helmholtz datasets;
finalize CRAN release of QresR; publish white paper/guidelines.

\subsection{Other aspects}\label{other-aspects}

Licensing: Code under MIT; data standards and metadata under CC-BY 4.0.

Hosting: Public GitHub repository with issue tracking and contribution
guidelines; Shiny app hosted on University of Cologne's R Shiny server.

Dissemination: 1. Announcement blog post on R Consortium blog at project
start. 2. Quarterly blog updates on progress. 3. Delivery blog post on
release of QresR (CRAN). 4. Presentations at UseR! 2026, ISC meetings,
and relevant microbiology conferences. 5. Social media announcements
(via CRC1310, FiMAR, and partner institutes).

\subsection{Budget \& funding plan}\label{budget-funding-plan}

Funding is requested primarily for labor costs: Two PhD students (12
months each) 1. PhD student 1 (Software Development \& Data
Engineering): Responsible for database, API, and storage pipelines. 2.
PhD student 2 (R/Shiny/API Development): Responsible for R package,
Shiny interface, and automated data submission. 3. The PI (Viera
Kováčová, University of Cologne) will supervise the project but does not
request salary support. 4. No funds are requested for travel,
publication fees, or hardware. University of Cologne provides server
infrastructure.

\begin{longtable}[]{@{}
  >{\raggedright\arraybackslash}p{(\linewidth - 8\tabcolsep) * \real{0.5789}}
  >{\raggedright\arraybackslash}p{(\linewidth - 8\tabcolsep) * \real{0.0316}}
  >{\raggedright\arraybackslash}p{(\linewidth - 8\tabcolsep) * \real{0.0947}}
  >{\raggedright\arraybackslash}p{(\linewidth - 8\tabcolsep) * \real{0.1789}}
  >{\raggedright\arraybackslash}p{(\linewidth - 8\tabcolsep) * \real{0.1158}}@{}}
\toprule\noalign{}
\begin{minipage}[b]{\linewidth}\raggedright
Role
\end{minipage} & \begin{minipage}[b]{\linewidth}\raggedright
FTE
\end{minipage} & \begin{minipage}[b]{\linewidth}\raggedright
Duration
\end{minipage} & \begin{minipage}[b]{\linewidth}\raggedright
Cost per year (€)
\end{minipage} & \begin{minipage}[b]{\linewidth}\raggedright
Total (€)
\end{minipage} \\
\midrule\noalign{}
\endhead
\bottomrule\noalign{}
\endlastfoot
PhD Student 1 -- Software Development \& Data Engineering & 1.0 & 12
months & 76,500 & 76,500 \\
PhD Student 2 -- R/Shiny/API Development & 1.0 & 12 months & 76,500 &
76,500 \\
Total Personnel & 2.0 & 12 months & -- & 153,000 \\
\end{longtable}

Milestones tied to funding release: Month 4: API + database schema
complete → €30,600 Month 6: Alpha R package + API documentation →
€30,600 Month 8: Shiny submission system + visualization → €30,600 Month
10: Integration with external tools + DGrowthR → €30,600 Month 12: CRAN
release, guidelines, dissemination → €30,600

\section{Budget Justification}\label{budget-justification}

The requested funding is dedicated entirely to two full-time PhD student
positions for 12 months. One PhD student will focus on software
development and data engineering (database design, storage pipelines,
REST API), while the second will focus on R/Shiny and API development (R
package, submission interface, visualization). This division of labor is
essential to ensure rapid and parallel progress across the backend and
user-facing components of Qres. The PI (Viera Kováčová) will supervise
the project but does not request salary support, ensuring that 100\% of
ISC funds are invested directly into infrastructure development for the
R community.

\section{Success}\label{success}

\subsection{Definition of done}\label{definition-of-done}

The project will be considered complete and successful when: 1. The Qres
database and REST API are fully operational, allowing storage and
retrieval of dose--response datasets with curated metadata. 2. The QresR
package is publicly released on CRAN and GitHub, enabling R users to
programmatically query, preprocess, and analyze datasets. 3. The Shiny
interface supports automated data submission (with validation) and
interactive download/visualization of raw data and fitted models. 4.
Integration with at least one external predictive tool (Python/Julia)
and DGrowthR is demonstrated through vignettes and tutorials. 5.
Community guidelines / white paper on best practices for dose--response
experiments are published and disseminated.

\subsection{Measuring success}\label{measuring-success}

Success will be measured using concrete, trackable outputs: 1. Technical
outputs: * QresR package accepted on CRAN. * API documentation publicly
available. * Shiny interface deployed with submission/download
capability.

\begin{enumerate}
\def\labelenumi{\arabic{enumi}.}
\setcounter{enumi}{1}
\tightlist
\item
  Community uptake:
\end{enumerate}

\begin{itemize}
\tightlist
\item
  At least 5 external datasets submitted and validated through the
  automated submission system.
\item
  Engagement of early adopters from CRC1310, FiMAR, Human Technopol, and
  Helmholtz.
\end{itemize}

\begin{enumerate}
\def\labelenumi{\arabic{enumi}.}
\setcounter{enumi}{2}
\tightlist
\item
  Dissemination:
\end{enumerate}

\begin{itemize}
\tightlist
\item
  Release of at least 2 vignettes/tutorials demonstrating workflows in
  R.
\item
  Publication of guidelines for dose--response data management and
  modeling.
\end{itemize}

\subsection{Future work}\label{future-work}

Once Qres reaches full functionality, we will prepare a high-impact
publication describing its architecture, features, and scientific use
cases. The manuscript will highlight: * The need for standardized,
FAIR-compliant dose--response data. * Qres infrastructure (database,
API, R package, Shiny interface). * Integration of predictive tools
across R, Python, and Julia. * Demonstration datasets from microbial
resistance and cancer evolution. * Benchmarks comparing automated
analysis pipelines (e.g.~DGrowthR, Kinbiont).

Target journals could include Nature Biotechnology, Nature
Communications, Nucleic Acids Research (Database issue), or
Bioinformatics.

This publication will provide a citable reference for Qres, ensuring
that dataset contributors receive recognition and that the platform is
embedded in the global research ecosystem.

\phantomsection\label{refs}
\begin{CSLReferences}{1}{0}
\bibitem[\citeproctext]{ref-buxfcrkner2017}
Bürkner, Paul-Christian. 2017. {``Brms: An r Package for Bayesian
Multilevel Models Using Stan.''} \emph{Journal of Statistical Software}
80 (1): 1--28.

\bibitem[\citeproctext]{ref-feldl2025}
Feldl, Medina, and Roberto Olayo Alarcon. 2025. \emph{DGrowthR:
Bacterial Growth Curve Analysis}.
\url{https://bio-datascience.github.io/DGrowthR/}.

\bibitem[\citeproctext]{ref-pinheiro2023}
Pinheiro, José, and Douglas Bates. 2023. \emph{Mixed-Effects Models in s
and s-PLUS}. Springer.

\bibitem[\citeproctext]{ref-ritz2015}
Ritz, Christian, Florent Baty, Jens C Streibig, and Daniel Gerhard.
2015. {``Dose-Response Analysis Using r.''} \emph{PLOS ONE} 10 (12):
e0146021.

\end{CSLReferences}




\end{document}
