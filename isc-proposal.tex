% Options for packages loaded elsewhere
% Options for packages loaded elsewhere
\PassOptionsToPackage{unicode}{hyperref}
\PassOptionsToPackage{hyphens}{url}
\PassOptionsToPackage{dvipsnames,svgnames,x11names}{xcolor}
%
\documentclass[
  letterpaper,
  DIV=11,
  numbers=noendperiod]{scrartcl}
\usepackage{xcolor}
\usepackage{amsmath,amssymb}
\setcounter{secnumdepth}{-\maxdimen} % remove section numbering
\usepackage{iftex}
\ifPDFTeX
  \usepackage[T1]{fontenc}
  \usepackage[utf8]{inputenc}
  \usepackage{textcomp} % provide euro and other symbols
\else % if luatex or xetex
  \usepackage{unicode-math} % this also loads fontspec
  \defaultfontfeatures{Scale=MatchLowercase}
  \defaultfontfeatures[\rmfamily]{Ligatures=TeX,Scale=1}
\fi
\usepackage{lmodern}
\ifPDFTeX\else
  % xetex/luatex font selection
\fi
% Use upquote if available, for straight quotes in verbatim environments
\IfFileExists{upquote.sty}{\usepackage{upquote}}{}
\IfFileExists{microtype.sty}{% use microtype if available
  \usepackage[]{microtype}
  \UseMicrotypeSet[protrusion]{basicmath} % disable protrusion for tt fonts
}{}
\makeatletter
\@ifundefined{KOMAClassName}{% if non-KOMA class
  \IfFileExists{parskip.sty}{%
    \usepackage{parskip}
  }{% else
    \setlength{\parindent}{0pt}
    \setlength{\parskip}{6pt plus 2pt minus 1pt}}
}{% if KOMA class
  \KOMAoptions{parskip=half}}
\makeatother
% Make \paragraph and \subparagraph free-standing
\makeatletter
\ifx\paragraph\undefined\else
  \let\oldparagraph\paragraph
  \renewcommand{\paragraph}{
    \@ifstar
      \xxxParagraphStar
      \xxxParagraphNoStar
  }
  \newcommand{\xxxParagraphStar}[1]{\oldparagraph*{#1}\mbox{}}
  \newcommand{\xxxParagraphNoStar}[1]{\oldparagraph{#1}\mbox{}}
\fi
\ifx\subparagraph\undefined\else
  \let\oldsubparagraph\subparagraph
  \renewcommand{\subparagraph}{
    \@ifstar
      \xxxSubParagraphStar
      \xxxSubParagraphNoStar
  }
  \newcommand{\xxxSubParagraphStar}[1]{\oldsubparagraph*{#1}\mbox{}}
  \newcommand{\xxxSubParagraphNoStar}[1]{\oldsubparagraph{#1}\mbox{}}
\fi
\makeatother


\usepackage{longtable,booktabs,array}
\usepackage{calc} % for calculating minipage widths
% Correct order of tables after \paragraph or \subparagraph
\usepackage{etoolbox}
\makeatletter
\patchcmd\longtable{\par}{\if@noskipsec\mbox{}\fi\par}{}{}
\makeatother
% Allow footnotes in longtable head/foot
\IfFileExists{footnotehyper.sty}{\usepackage{footnotehyper}}{\usepackage{footnote}}
\makesavenoteenv{longtable}
\usepackage{graphicx}
\makeatletter
\newsavebox\pandoc@box
\newcommand*\pandocbounded[1]{% scales image to fit in text height/width
  \sbox\pandoc@box{#1}%
  \Gscale@div\@tempa{\textheight}{\dimexpr\ht\pandoc@box+\dp\pandoc@box\relax}%
  \Gscale@div\@tempb{\linewidth}{\wd\pandoc@box}%
  \ifdim\@tempb\p@<\@tempa\p@\let\@tempa\@tempb\fi% select the smaller of both
  \ifdim\@tempa\p@<\p@\scalebox{\@tempa}{\usebox\pandoc@box}%
  \else\usebox{\pandoc@box}%
  \fi%
}
% Set default figure placement to htbp
\def\fps@figure{htbp}
\makeatother


% definitions for citeproc citations
\NewDocumentCommand\citeproctext{}{}
\NewDocumentCommand\citeproc{mm}{%
  \begingroup\def\citeproctext{#2}\cite{#1}\endgroup}
\makeatletter
 % allow citations to break across lines
 \let\@cite@ofmt\@firstofone
 % avoid brackets around text for \cite:
 \def\@biblabel#1{}
 \def\@cite#1#2{{#1\if@tempswa , #2\fi}}
\makeatother
\newlength{\cslhangindent}
\setlength{\cslhangindent}{1.5em}
\newlength{\csllabelwidth}
\setlength{\csllabelwidth}{3em}
\newenvironment{CSLReferences}[2] % #1 hanging-indent, #2 entry-spacing
 {\begin{list}{}{%
  \setlength{\itemindent}{0pt}
  \setlength{\leftmargin}{0pt}
  \setlength{\parsep}{0pt}
  % turn on hanging indent if param 1 is 1
  \ifodd #1
   \setlength{\leftmargin}{\cslhangindent}
   \setlength{\itemindent}{-1\cslhangindent}
  \fi
  % set entry spacing
  \setlength{\itemsep}{#2\baselineskip}}}
 {\end{list}}
\usepackage{calc}
\newcommand{\CSLBlock}[1]{\hfill\break\parbox[t]{\linewidth}{\strut\ignorespaces#1\strut}}
\newcommand{\CSLLeftMargin}[1]{\parbox[t]{\csllabelwidth}{\strut#1\strut}}
\newcommand{\CSLRightInline}[1]{\parbox[t]{\linewidth - \csllabelwidth}{\strut#1\strut}}
\newcommand{\CSLIndent}[1]{\hspace{\cslhangindent}#1}



\setlength{\emergencystretch}{3em} % prevent overfull lines

\providecommand{\tightlist}{%
  \setlength{\itemsep}{0pt}\setlength{\parskip}{0pt}}



 


\KOMAoption{captions}{tableheading}
\makeatletter
\@ifpackageloaded{caption}{}{\usepackage{caption}}
\AtBeginDocument{%
\ifdefined\contentsname
  \renewcommand*\contentsname{Table of contents}
\else
  \newcommand\contentsname{Table of contents}
\fi
\ifdefined\listfigurename
  \renewcommand*\listfigurename{List of Figures}
\else
  \newcommand\listfigurename{List of Figures}
\fi
\ifdefined\listtablename
  \renewcommand*\listtablename{List of Tables}
\else
  \newcommand\listtablename{List of Tables}
\fi
\ifdefined\figurename
  \renewcommand*\figurename{Figure}
\else
  \newcommand\figurename{Figure}
\fi
\ifdefined\tablename
  \renewcommand*\tablename{Table}
\else
  \newcommand\tablename{Table}
\fi
}
\@ifpackageloaded{float}{}{\usepackage{float}}
\floatstyle{ruled}
\@ifundefined{c@chapter}{\newfloat{codelisting}{h}{lop}}{\newfloat{codelisting}{h}{lop}[chapter]}
\floatname{codelisting}{Listing}
\newcommand*\listoflistings{\listof{codelisting}{List of Listings}}
\makeatother
\makeatletter
\makeatother
\makeatletter
\@ifpackageloaded{caption}{}{\usepackage{caption}}
\@ifpackageloaded{subcaption}{}{\usepackage{subcaption}}
\makeatother
\usepackage{bookmark}
\IfFileExists{xurl.sty}{\usepackage{xurl}}{} % add URL line breaks if available
\urlstyle{same}
\hypersetup{
  pdftitle={QRes - An Open-Source Dose--Response Data Platform for the R Community},
  pdfauthor={Viera Kovacova},
  colorlinks=true,
  linkcolor={blue},
  filecolor={Maroon},
  citecolor={Blue},
  urlcolor={Blue},
  pdfcreator={LaTeX via pandoc}}


\title{QRes - An Open-Source Dose--Response Data Platform for the R
Community}
\author{Viera Kovacova}
\date{2025-09-30}
\begin{document}
\maketitle


\section{Executive Summary}\label{executive-summary}

Dose--response data are fundamental to research in microbiology
(Lukačišinová, Fernando, and Bollenbach 2020), oncology (Kuosmanen
2021), pharmacology (Bretz, Bornkamp, and Dumortier 2025; Wang et al.
2020), biophysics (Petrungaro et al. 2025), and toxicology (Lukačišin
and Bollenbach 2019). Yet raw datasets remain fragmented, inconsistently
formatted, and often lack metadata, making them difficult to integrate
into reproducible R workflows. Frequently, raw measurements are not
shared at all, with only fitted curves or summary metrics (such as
IC₅₀/OD₅₀) reported---sometimes based on methods inappropriate for the
underlying data (Bayer et al. 2023; Wang et al. 2020).

In practice, raw dose--response data are stored in \textbf{small
plain-text files reporting time and measured values}. Even for
large-scale, parallelized experiments recorded on robotic systems over
several weeks, these files rarely exceed \textbf{1 MB in size}, making
them highly portable and easy to archive. When available, raw
time-resolved growth data enable robust model benchmarking, transparent
comparison of analytical methods, and reproducibility across studies.
Beyond replication, curated raw datasets open the door to predictive
modeling of evolutionary dynamics and treatment outcomes (Kuosmanen
2021; Angaroni 2025), offering a unique opportunity to accelerate both
fundamental discovery and translational applications.

The \href{https://growthcurvedb.shinyapps.io/growthcurvedb/}{Qres
platform} is an open-access dose--response database and web service
hosted at the \href{https://crc1310.uni-koeln.de/}{University of
Cologne}. Qres is designed to transform how raw microbial and cancer
cell growth curve data are collected, curated, and analyzed. By
centralizing access and standardizing analysis, Qres will speed up
progress in understanding resistance dynamics and cellular responses
under perturbation. This proposal requests R Consortium support to
expand Qres from its current use within
\href{https://crc1310.uni-koeln.de/}{CRC1310: Predictability in
Evolution} to a fully R-integrated, open-source resource accessible to
the global R community and beyond.

The project will deliver:\\
- A \textbf{public API} and an \textbf{R package} (\texttt{QresR}) for
standardized data access, database storage, and analysis.\\
- An \textbf{enhanced R Shiny interface} to upload and download raw data
with metadata, validate new datasets, and enable interactive
visualization and analysis directly in the browser.\\
- \textbf{Integration of predictive tools} developed in Python by
\href{https://www.fimar.fi/}{Prof.~Ville Mustonen, FiMAR} and in Julia
by \href{https://humantechnopole.it/en/}{Dr.~Fernanda Pinheiro, Human
Technopole, Milan}, enabling cross-language interoperability.\\
- Incorporation of the
\href{https://bio-datascience.github.io/DGrowthR/}{DGrowthR package}
(Feldl and Olayo Alarcon 2025) developed at
\href{https://www.helmholtz-munich.de/en}{Helmholtz Munich} as an
analysis option for bacterial growth data.

Key outputs include an R package released on CRAN, standardized data
formats, reproducible workflows, and community guidelines establishing
\textbf{best practices and gold standards} for dose--response
experiments. All analytical tools developed within Qres will be
\textbf{species agnostic}, ensuring applicability across microbes,
cancer models, pharmacological assays, and toxicology screens. By
bridging R with Python and Julia tools, Qres will provide a foundation
for AI-driven analysis and long-term reproducible science. This project
directly addresses the R Consortium's mission by strengthening technical
infrastructure, fostering interoperability, and serving a broad
international user base across biology, biophysics, and medicine.

The requested budget of \textbf{€50,000} will be used entirely for
direct project development: a part-time developer for database and API
design, a student assistant for metadata curation and testing, limited
infrastructure costs, and targeted consultancy to optimize code and
workflows. The PI does not request salary support, ensuring that all
funds directly advance infrastructure for the R community.

\section{Signatories}\label{signatories}

This section summarizes the individuals and groups who support,
contribute to, or have been consulted on the Qres proposal. The project
is already embedded in large research networks:
\href{https://crc1310.uni-koeln.de/}{CRC1310: Predictability in
Evolution, University of Cologne}, \href{https://www.fimar.fi/}{FiMAR:
Finnish Centre of Excellence in Antimicrobial Resistance Research},
\href{https://humantechnopole.it/en/}{Human Technopole, Milan},
\href{https://crc1678.uni-koeln.de/}{CRC1678: Systems-level consequences
of fidelity changes in mRNA and protein biosynthesis},
\href{https://www.helmholtz-munich.de/en}{Helmholtz Munich}).

\subsection{Project team}\label{project-team}

\begin{itemize}
\tightlist
\item
  \textbf{Viera Kováčová, PhD}
  (\href{mailto:viera.kovacova@uni-koeln.de}{\nolinkurl{viera.kovacova@uni-koeln.de}})
  -- Principal Investigator. Leads the development of Qres, supervises
  technical and scientific progress, and coordinates collaborations.\\
\item
  \textbf{Prof.~Michael Lässig}
  (\href{mailto:lassig@uni-koeln.de}{\nolinkurl{lassig@uni-koeln.de}})
  -- Director of CRC1310. Provides strategic oversight and advice on the
  broader scientific vision.\\
\item
  \textbf{Stephan Kleinboltig}
  (\href{mailto:kleinboltig@uni-koeln.de}{\nolinkurl{kleinboltig@uni-koeln.de}})
  -- CRC1310 project manager. Supports administrative coordination and R
  server infrastructure at the University of Cologne.\\
\item
  \textbf{Part-time Developer (0.5 FTE, to be hired)} -- Specialist in
  database and API development. Will co-design the Qres database schema,
  implement the public API, and ensure efficient system architecture.\\
\item
  \textbf{Student Assistant (Data Steward, to be hired)} -- Supports
  metadata curation, testing of submission workflows, and validation of
  Shiny interface features.
\end{itemize}

\subsection{Contributors}\label{contributors}

\begin{itemize}
\tightlist
\item
  Prof.~Ville Mustonen (FiMAR, Helsinki) -- Provides predictive modeling
  expertise, helps shape the scientific scope, and engages international
  collaborators to contribute datasets.
\item
  Dr.~Fernanda Pinheiro (Human Technopol, Milan) -- Leads development of
  the Kinbiont Julia package for bacterial growth modeling; collaborates
  on integration into Qres.
\item
  Prof.~Tobias Bollenbach (CRC1310, University of Cologne) -- PI
  producing large-scale dose--response datasets; supports data
  contribution and benchmarking.
\end{itemize}

\subsection{Consulted}\label{consulted}

\begin{enumerate}
\def\labelenumi{\arabic{enumi}.}
\item
  \textbf{Colleagues and mentors} -- contributing feedback, expert
  opinion, and testing:

  \begin{itemize}
  \tightlist
  \item
    Theresa Finke, PhD candidate, University of Cologne
  \item
    Leon Sieger, PhD candidate, University of Cologne
  \item
    Dr.~Gabriela Pertungaro, University of Cologne (Petrungaro et al.
    2025)
  \item
    Dr.~Rotem Gross (Gross et al. 2024)
  \item
    Dr.~Fabrizio Angaroni (Angaroni 2025)
  \item
    Medina Feldl, PhD candidate, Helmholtz, Munich -- Developer of the
    DGrowthR package (Feldl and Olayo Alarcon 2025); contributes to Qres
    integration for bacterial dose--response analysis.
  \item
    Prof.~Andreas Beyer
    (\href{mailto:andreas.beyer@uni-koeln.de}{\nolinkurl{andreas.beyer@uni-koeln.de}})
    -- Director of CRC1678; advises on systems biology integration and
    data-driven approaches.
  \item
    Prof.~Miroslav Baránek (Mendel University, Lednice, Czech Republic)
    -- Provides perspective on dose--response measurements in plant
    pathogens.
  \end{itemize}
\item
  \textbf{C3RDM and ITCC departments, University of Cologne} --
  consulted on IT infrastructure, data management, and server support.
\item
  \textbf{Industry contacts} -- representatives from Agilent, Tecan,
  Molecular Devices, Promega, and BMGLabtech, who provided test/sample
  files from microplate readers and laboratory robots to inform data
  ingestion pipelines.
\end{enumerate}

\subsection{Community Support}\label{community-support}

The project is known and endorsed within CRC1310 (all PIs listed at
\href{crc1310.uni-koeln.de/project_a.html}{Area A}). This broad base of
support ensures that Qres is well-positioned for adoption and
sustainability, with immediate access to diverse microbial
dose--response datasets.

\section{The Problem}\label{the-problem}

Dose--response measurements are among the most fundamental experiments
in biology: they quantify how cells, microbes, and tissues respond to
drugs, antibiotics, toxins, or environmental perturbations. These data
are critical for understanding microbial resistance, cancer treatment
response, and pharmacological efficacy. Despite their central role, the
raw growth curve data underlying dose--response studies remain
fragmented, inconsistently reported, and difficult to reuse.

\subsection{Who it affects:}\label{who-it-affects}

\begin{enumerate}
\def\labelenumi{\arabic{enumi}.}
\tightlist
\item
  Experimental researchers face difficulties identifying optimal
  concentrations or conditions because raw data are buried in
  supplementary materials or inaccessible formats.
\item
  Modelers and computational biologists cannot easily benchmark or
  parameterize models due to lack of standardized, cross-study datasets.
\item
  The R community has strong modeling tools (e.g.~drc, nlme, brms) but
  lacks a central resource of curated datasets with reproducible access
  pathways.
\end{enumerate}

\subsection{Why it is a problem:}\label{why-it-is-a-problem}

\begin{enumerate}
\def\labelenumi{\arabic{enumi}.}
\tightlist
\item
  Lack of accessible raw data prevents reproducibility and cross-study
  comparison.
\item
  Metadata are often incomplete, making it impossible to assess
  experimental context and test the importance of parameters.
\item
  Absence of standards leads to inconsistent extraction of resistance
  traits and dose--response parameters.
\item
  Inappropriate or ad hoc model fitting is common, with no accepted
  standard for model selection or parameter estimation (Bayer et al.
  2023). This not only risks misinterpretation of experimental results
  but also reduces the comparability of datasets. Without broadly
  available raw data, it is impossible to build the large, standardized
  data lakes needed to train and validate AI/ML tools for optimized drug
  management and resistance prediction (Wang et al. 2020).
\end{enumerate}

These gaps collectively hinder reproducibility, slow down discovery, and
prevent the R community from becoming the central hub for dose--response
analysis.

\subsection{What solving the problem
enables:}\label{what-solving-the-problem-enables}

\begin{enumerate}
\def\labelenumi{\arabic{enumi}.}
\tightlist
\item
  Establishing the Qres platform will provide the R community with a
  curated, FAIR-compliant infrastructure for dose--response data.
\item
  Researchers will be able to upload and share raw datasets with
  metadata, ensuring proper citation and recognition.
\item
  Modelers will have standardized, comparable inputs for parameter
  estimation, benchmarking, and AI-driven predictions.
\item
  R users will benefit from integrated pipelines linking raw data to
  statistical and mechanistic modeling.
\item
  For society, Qres will deepen our understanding of microbial
  resistance and its evolution, support better-informed strategies to
  combat antibiotic and drug resistance, and provide benchmarks that
  improve the design of therapeutic interventions. By enabling
  systematic cross-study comparisons, Qres will help identify resistance
  patterns earlier, guide clinical and experimental decisions, and
  ultimately contribute to public health.
\end{enumerate}

\subsection{Existing work:}\label{existing-work}

\begin{enumerate}
\def\labelenumi{\arabic{enumi}.}
\tightlist
\item
  R packages such as \textbf{drc} (Ritz et al. 2015), \textbf{nlme}
  (Pinheiro and Bates 2023), and \textbf{brms} (Bürkner 2017) support
  flexible model fitting, but they assume users already work with
  curated datasets and provide no infrastructure for raw data
  management.
\item
  \textbf{DGrowthR} (Helmholtz Munich) (Feldl and Olayo Alarcon 2025)
  enables bacterial growth curve analysis in R, but does not provide
  access to standardized raw data repositories.
\item
  \textbf{Kinbiont} (Julia) (Angaroni 2025) implements growth-curve
  fitting and predictive modeling, and \textbf{bmdrc} (Python) (Degnan
  2025) provides benchmark dose--response analysis, but both are
  standalone packages without direct integration into R workflows.
\item
  Foundational models of microbial growth go back to \textbf{Monod
  (1949)} (Monod 1949), underscoring that dose--response analysis has
  been a central theme in biology for decades. Yet despite this long
  history, there is still no shared platform for raw data curation,
  standardized metadata, and interoperable analytical pipelines.
\end{enumerate}

To date, there is no centralized, open-access platform in R that
combines data storage, metadata standards, reproducible access, and
interactive visualization. This gap limits reproducibility, slows the
pace of discovery, and prevents the R community from fully supporting
the next generation of data-driven microbiology and pharmacology.

\section{The proposal}\label{the-proposal}

\subsection{Approach and Work Plan}\label{approach-and-work-plan}

We will address the problem of fragmented and inaccessible
dose--response datasets by developing Qres into an open, standardized,
R-integrated platform with the following concrete actions:

\subsubsection{API and Database Development (Months
1--4)}\label{api-and-database-development-months-14}

\begin{itemize}
\tightlist
\item
  \textbf{Part-time developer (0.5 FTE):} implement a REST API with
  standardized JSON schema, under guidance from the PI and with
  consultant code review.
\item
  \textbf{Student assistant:} begin collecting and formatting test
  datasets with metadata for ingestion.
\item
  Expand the database schema to store raw growth curves and curated
  metadata.
\item
  Establish FAIR-compliant pipelines for data ingestion.
\end{itemize}

\subsubsection{R Package (Months 5--6)}\label{r-package-months-56}

\begin{itemize}
\tightlist
\item
  \textbf{PI:} develops the open-source R package (QresR) with support
  from the part-time developer.
\item
  Provide functions for data preprocessing, extraction of resistance
  traits, and integration with existing modeling packages (\emph{drc},
  \emph{nlme}, \emph{brms}).
\item
  Begin integration of \textbf{DGrowthR} for bulk growth curve analysis.
\end{itemize}

\subsubsection{Shiny Interface and Submission Form (Months
7--8)}\label{shiny-interface-and-submission-form-months-78}

\begin{itemize}
\tightlist
\item
  \textbf{PI:} leads development of the Shiny modules, with the student
  assistant testing and curating submissions.
\item
  Extend the existing Shiny prototype into a full interface supporting
  upload and download of raw growth data with metadata.
\item
  Implement automated submission validation and error reporting.
\item
  Add interactive visualization and curve-fitting modules.
\end{itemize}

\subsubsection{Cross-Language Integration (Months
9--10)}\label{cross-language-integration-months-910}

\begin{itemize}
\tightlist
\item
  Reimplement or wrap selected Python/Julia predictive tools developed
  by collaborators (Mustonen, Pinheiro).
\item
  Benchmark results across platforms to ensure reproducibility.
\end{itemize}

\subsubsection{Testing, Standards, and Dissemination (Months
11--12)}\label{testing-standards-and-dissemination-months-1112}

\begin{itemize}
\tightlist
\item
  Pilot test data submissions with internal and external collaborators.
\item
  Optimize submission form and analysis workflows.
\item
  Publish the R package on CRAN, complete documentation, tutorials, and
  case studies.
\item
  Release community guidelines establishing gold standards for
  dose--response experiments.
\end{itemize}

\textbf{Lean development model:} The PI (Viera Kováčová) will co-develop
the R/Shiny code, supported by a part-time developer (0.5 FTE)
responsible for database and API implementation, and a student assistant
acting as data steward for metadata curation and testing. External
consultants will provide targeted mentoring and code review to optimize
workflows.

\begin{center}\rule{0.5\linewidth}{0.5pt}\end{center}

\subsection{Risk Management}\label{risk-management}

\begin{enumerate}
\def\labelenumi{\arabic{enumi}.}
\tightlist
\item
  \textbf{Data heterogeneity} → Mitigation: enforce a standardized
  metadata schema and automated validation in the Shiny submission form.
\item
  \textbf{Integration complexity of Python/Julia tools} → Mitigation:
  begin with a small subset of methods and provide R wrappers before
  attempting full reimplementations, with consultant support.
\item
  \textbf{Limited adoption} → Mitigation: leverage
  \href{https://crc1310.uni-koeln.de/}{CRC1310},
  \href{https://www.fimar.fi/}{FiMAR}, and
  \href{https://humantechnopole.it/en/}{Human Technopole} networks to
  ensure early use and dataset contributions; provide persistent
  identifiers for proper citation.
\end{enumerate}

\begin{center}\rule{0.5\linewidth}{0.5pt}\end{center}

\subsection{Timeline}\label{timeline}

\begin{longtable}[]{@{}
  >{\raggedright\arraybackslash}p{(\linewidth - 2\tabcolsep) * \real{0.2778}}
  >{\raggedright\arraybackslash}p{(\linewidth - 2\tabcolsep) * \real{0.7222}}@{}}
\toprule\noalign{}
\begin{minipage}[b]{\linewidth}\raggedright
Months
\end{minipage} & \begin{minipage}[b]{\linewidth}\raggedright
Milestones
\end{minipage} \\
\midrule\noalign{}
\endhead
\bottomrule\noalign{}
\endlastfoot
1--4 & Database schema finalized; REST API prototype; test datasets
curated; FAIR-compliant ingestion pipeline established. \\
5--6 & QresR R package prototype released on GitHub; functions for data
access and preprocessing; begin DGrowthR integration. \\
7--8 & Shiny submission system extended: upload/download with metadata
validation; visualization and curve-fitting modules implemented. \\
9--10 & Cross-language integration: selected Python (FiMAR) and Julia
(Human Technopole) tools wrapped/reimplemented in R; reproducibility
benchmarking. \\
11 & Pilot testing with CRC1310, FiMAR, and Helmholtz datasets;
optimization of submission workflows. \\
12 & CRAN release of QresR; full documentation, tutorials, case studies;
community guidelines published. \\
\end{longtable}

\begin{center}\rule{0.5\linewidth}{0.5pt}\end{center}

\subsection{Overview}\label{overview}

The Qres platform is an open-access dose--response database and web
service that collects, organizes, and analyzes raw, time-resolved
microbial growth curves under antibiotic, genetic, and environmental
perturbations. Currently, raw dose--response data are fragmented,
inconsistently reported, and often hidden in supplementary files, making
it difficult for researchers to reproduce analyses, perform cross-study
comparisons, or build robust models of microbial resistance and cancer
evolution.

Qres directly addresses this problem by providing the R community with a
standardized, open-source infrastructure for dose--response data. The
platform combines a robust database and public API, an R package (QresR)
for reproducible workflows, and an extended Shiny interface that enables
both upload and download of raw data with curated metadata, along with
interactive visualization and analysis.

The project is developed under a \textbf{lean staffing plan} (PI +
part-time developer + student assistant, plus targeted consultancy) and
in collaboration with international partners:\\
- Predictive and analytical tools from
\href{https://www.fimar.fi/}{FiMAR, Helsinki} and
\href{https://humantechnopole.it/en/}{Human Technopole, Milan} will be
reimplemented or wrapped for integration into R.\\
- The \href{https://bio-datascience.github.io/DGrowthR/}{DGrowthR
package} (Feldl and Olayo Alarcon 2025) (Helmholtz, Munich) will be
incorporated for bacterial growth curve analysis.

\textbf{Benefits to the R community:}

\begin{itemize}
\tightlist
\item
  Access to curated, FAIR-compliant raw dose--response datasets.
\item
  Reproducible workflows for microbial growth, cancer cell growth, and
  resistance modeling.
\item
  Integration of cross-language predictive tools into R.
\item
  Community guidelines establishing gold standards for dose--response
  data.
\end{itemize}

By solving the accessibility and reproducibility gap in dose--response
research, Qres will accelerate discovery, improve experimental design,
and expand R's role as the central platform for quantitative biology
(e.g., antibiotic resistance research).

\begin{center}\rule{0.5\linewidth}{0.5pt}\end{center}

\subsection{Detail}\label{detail}

\subsubsection{Minimum Viable Product}\label{minimum-viable-product}

The smallest version of Qres that delivers value to the R community will
include:\\
1. A database + REST API for storing and retrieving standardized
dose--response datasets.\\
2. An R package (QresR) providing functions to query the API and return
data in tidy formats for downstream analysis with packages like
\emph{drc}, \emph{nlme}, and \emph{brms}.\\
3. A basic Shiny interface that allows users to upload raw growth curve
data with metadata, validates submissions, and enables download of
standardized datasets.

This MVP ensures that from the beginning, R users can both contribute
new datasets and reuse existing data in reproducible workflows.

\subsubsection{Architecture}\label{architecture}

At a high level, the architecture will consist of:\\
1. \textbf{Database Layer} -- stores raw growth measurements, curated
metadata, and derived resistance traits; implements FAIR-compliant
schemas with persistent identifiers.\\
2. \textbf{API Layer} -- RESTful API (JSON-based) for standardized
programmatic access; enables both raw data retrieval and derived
parameter queries.\\
3. \textbf{R Package (QresR)} -- provides functions to query the API,
clean and preprocess data, and connect to modeling packages; interfaces
with external tools such as DGrowthR.\\
4. \textbf{Web Interface (Shiny)} -- supports upload of raw data +
metadata via automated submission form; performs validation, error
reporting, and visualization of fitted curves; allows users to explore
and download datasets interactively.\\
5. \textbf{External Tool Integration} -- wrappers or reimplementations
for Python/Julia

\section{Project plan}\label{project-plan}

\subsection{Start-up phase}\label{start-up-phase}

Collaboration platform: We will host all code on GitHub under an open
repository (qres-platform), including the API, Shiny interface, and R
package. Issues and pull requests will be used to manage contributions
from collaborators and the community.

License decisions: All code will be released under the MIT license; data
schemas and metadata standards will be released under CC-BY 4.0 to
maximize reuse.

Reporting framework: We will provide quarterly progress updates to the
ISC, including blog posts summarizing milestones. The GitHub repository
will include a project board to track deliverables.

\subsection{Technical delivery}\label{technical-delivery}

The project will be delivered in phased milestones over 12 months, with
work shared between the PI (R/Shiny development), a part-time developer
(database and API expertise), a student assistant (data curation and
testing), and external consultants (mentorship and optimization).

\begin{itemize}
\tightlist
\item
  \textbf{Months 1--2:}

  \begin{itemize}
  \tightlist
  \item
    Hire part-time developer and student assistant.
  \item
    Define database schema and metadata standards.
  \item
    Set up GitHub repository with open license and contribution
    guidelines.
  \item
    Begin API blueprint under developer's guidance.
  \end{itemize}
\item
  \textbf{Months 3--4:}

  \begin{itemize}
  \tightlist
  \item
    Developer: implement core database backend and initial API
    endpoints.
  \item
    PI: create skeleton of the QresR package to connect to API.
  \item
    Student assistant: curate test datasets and metadata.
  \item
    First internal prototype of Shiny data upload form.
  \end{itemize}
\item
  \textbf{Months 5--6:}

  \begin{itemize}
  \tightlist
  \item
    Release alpha version of QresR package on GitHub.
  \item
    Developer: extend API functionality and documentation.
  \item
    PI: integrate Shiny form with API for validation.
  \item
    Student assistant: test metadata submission workflows.
  \end{itemize}
\item
  \textbf{Months 7--8:}

  \begin{itemize}
  \tightlist
  \item
    Extend Shiny interface with interactive visualization modules.
  \item
    Enable validated external test submissions.
  \item
    Consultants: provide code review and mentoring on API efficiency and
    R optimization.
  \end{itemize}
\item
  \textbf{Months 9--10:}

  \begin{itemize}
  \tightlist
  \item
    Integrate DGrowthR workflows for bulk growth analysis.
  \item
    Provide R wrappers for selected predictive tools (Python/Julia) with
    consultant guidance.
  \item
    Student assistant: continue testing with real CRC1310/FiMAR
    datasets.
  \end{itemize}
\item
  \textbf{Months 11--12:}

  \begin{itemize}
  \tightlist
  \item
    Community testing with CRC1310, FiMAR, and Helmholtz partners.
  \item
    Finalize CRAN release of QresR package with vignettes.
  \item
    Publish guidelines/white paper on dose--response data standards.
  \item
    Disseminate results via R Consortium blog and community channels.
  \end{itemize}
\end{itemize}

\subsection{Other aspects}\label{other-aspects}

Licensing: Code under MIT; data standards and metadata under CC-BY 4.0.

Hosting: Public GitHub repository with issue tracking and contribution
guidelines; Shiny app hosted on University of Cologne's R Shiny server.

Dissemination:

\begin{enumerate}
\def\labelenumi{\arabic{enumi}.}
\tightlist
\item
  Announcement blog post on R Consortium blog at project start.
\item
  Quarterly blog updates on progress.
\item
  Delivery blog post on release of QresR (CRAN).
\item
  Presentations at UseR! 2026, ISC meetings, and relevant microbiology
  and drug resistance conferences.
\item
  Social media announcements (via CRC1310, FiMAR, and partner
  institutes).
\end{enumerate}

\section{Budget \& Funding Plan}\label{budget-funding-plan}

\subsection{Budget Justification}\label{budget-justification}

The requested funding will be used entirely for direct project
development, with a focus on delivering a functional, open-source Qres
platform within 12 months. We propose a lean model: a part-time
developer to co-implement the database and API, a student assistant to
support data stewardship and testing, and limited funds for
infrastructure and expert consultancy. The PI, Viera Kováčová, will lead
the project and contribute actively to programming (R/Shiny interface,
metadata submission form, R package), while also receiving mentorship
from the funded developer and consultants to strengthen her expertise in
database and API design. No PI salary support is requested.

\begin{longtable}[]{@{}
  >{\raggedright\arraybackslash}p{(\linewidth - 4\tabcolsep) * \real{0.3333}}
  >{\raggedright\arraybackslash}p{(\linewidth - 4\tabcolsep) * \real{0.3333}}
  >{\raggedright\arraybackslash}p{(\linewidth - 4\tabcolsep) * \real{0.3333}}@{}}
\toprule\noalign{}
\begin{minipage}[b]{\linewidth}\raggedright
Item
\end{minipage} & \begin{minipage}[b]{\linewidth}\raggedright
Estimated Cost (€)
\end{minipage} & \begin{minipage}[b]{\linewidth}\raggedright
Notes
\end{minipage} \\
\midrule\noalign{}
\endhead
\bottomrule\noalign{}
\endlastfoot
\textbf{Part-time Developer (0.5 FTE, 8--10 months)} & 32,000 & Skilled
in database/API development; designs backend, mentors PI on best
practices \\
\textbf{Student Assistant (Data Steward, 8h/week, 12 months)} & 6,000 &
Supports metadata curation, data validation, Shiny testing \\
\textbf{Miscellaneous / Infrastructure} & 2,000 & Server extensions,
additional storage, small-scale shinyapps.io environment for testing \\
\textbf{Consultancy / Mentorship} & 10,000 & Contracted expert(s) from
the R community for regular guidance on code optimization, database
design, and API best practices \\
\textbf{Total} & \textbf{50,000} & \\
\end{longtable}

\subsection{Milestones linked to
funding}\label{milestones-linked-to-funding}

Funding is requested primarily for labor costs:

\begin{longtable}[]{@{}
  >{\raggedright\arraybackslash}p{(\linewidth - 4\tabcolsep) * \real{0.3333}}
  >{\raggedright\arraybackslash}p{(\linewidth - 4\tabcolsep) * \real{0.3333}}
  >{\raggedright\arraybackslash}p{(\linewidth - 4\tabcolsep) * \real{0.3333}}@{}}
\toprule\noalign{}
\begin{minipage}[b]{\linewidth}\raggedright
Month(s)
\end{minipage} & \begin{minipage}[b]{\linewidth}\raggedright
Deliverable / Milestone
\end{minipage} & \begin{minipage}[b]{\linewidth}\raggedright
Funding (€)
\end{minipage} \\
\midrule\noalign{}
\endhead
\bottomrule\noalign{}
\endlastfoot
2 & Database schema + REST API prototype; initial test datasets curated
& 10,000 \\
4 & Extended database + FAIR ingestion pipeline; API documentation draft
& 10,000 \\
6 & QresR R package prototype on GitHub; DGrowthR integration started &
10,000 \\
8 & Shiny submission system with metadata validation + visualization
modules & 10,000 \\
12 & CRAN release of QresR; tutorials, case studies, community
guidelines & 10,000 \\
\textbf{Total} & & \textbf{50,000} \\
\end{longtable}

\textbf{Notes / Flexibility}

\begin{enumerate}
\def\labelenumi{\arabic{enumi}.}
\tightlist
\item
  The Consultancy/Mentorship line (€10k) can be adjusted downward
  (e.g.~€5--7k) if reviewers push back on budget.
\item
  University of Cologne will provide an additional student assistant
  (in-kind contribution, not funded by ISC).
\item
  No overhead, travel, or PI salary is requested.
\end{enumerate}

\section{Success}\label{success}

\subsection{Definition of done}\label{definition-of-done}

The project will be considered complete and successful when:

\begin{enumerate}
\def\labelenumi{\arabic{enumi}.}
\item
  The Qres database and REST API are fully operational, allowing storage
  and retrieval of dose--response datasets with curated metadata.
\item
  The QresR package is publicly released on CRAN and GitHub, enabling R
  users to programmatically query, preprocess, and analyze datasets.
\item
  The Shiny interface supports automated data submission (with
  validation) and interactive download/visualization of raw data and
  fitted models.
\item
  Integration with external predictive tools (Kinbiont Julia (Angaroni
  2025), DGrowthR (Feldl and Olayo Alarcon 2025) and \emph{bmdrc} Python
  (Degnan 2025)) is demonstrated through vignettes and tutorials.
\item
  Community guidelines / white paper on best practices for
  dose--response experiments are published and disseminated.
\end{enumerate}

\subsection{Measuring success}\label{measuring-success}

Success will be measured using concrete, trackable outputs:

\begin{enumerate}
\def\labelenumi{\arabic{enumi}.}
\tightlist
\item
  Technical outputs:
\end{enumerate}

\begin{itemize}
\tightlist
\item
  QresR package accepted on CRAN.
\item
  API documentation publicly available.
\item
  Shiny interface deployed with submission/download capability.
\end{itemize}

\begin{enumerate}
\def\labelenumi{\arabic{enumi}.}
\setcounter{enumi}{1}
\tightlist
\item
  Community uptake:
\end{enumerate}

\begin{itemize}
\item
  At least 5 external datasets submitted and validated through the
  automated submission system.
\item
  Engagement of early adopters from:

  \begin{itemize}
  \tightlist
  \item
    \href{https://crc1310.uni-koeln.de/}{CRC1310: Predictability in
    Evolution, University of Cologne}
  \item
    \href{https://www.fimar.fi/}{FiMAR: Finnish Centre of Excellence in
    Antimicrobial Resistance Research}
  \item
    \href{https://humantechnopole.it/en/}{Human Technopole, Milan}
  \item
    \href{https://www.helmholtz-munich.de/en}{Helmholtz Munich --
    Helmholtz Zentrum München}
  \end{itemize}
\end{itemize}

\begin{enumerate}
\def\labelenumi{\arabic{enumi}.}
\setcounter{enumi}{2}
\tightlist
\item
  Dissemination:
\end{enumerate}

\begin{itemize}
\tightlist
\item
  Release of at least 2 vignettes/tutorials demonstrating workflows in
  R.
\item
  Publication of guidelines for dose--response data management and
  modeling.
\end{itemize}

\subsection{Future work}\label{future-work}

Once Qres reaches full functionality, we will prepare a high-impact
publication describing its architecture, features, and scientific use
cases. The manuscript will highlight:

\begin{itemize}
\tightlist
\item
  The need for standardized, FAIR-compliant dose--response data.
\item
  Qres infrastructure (database, API, R package, Shiny interface).
\item
  Integration of predictive tools across R, Python, and Julia.
\item
  Demonstration datasets from microbial resistance and cancer evolution.
\item
  Benchmarks comparing automated analysis pipelines (e.g., DGrowthR,
  Kinbiont).
\end{itemize}

Target journals could include Nature Biotechnology, Nature
Communications, Nucleic Acids Research (Database issue), or
Bioinformatics. This publication will provide a citable reference for
Qres, ensuring that dataset contributors receive recognition and that
the platform is embedded in the global research ecosystem.

\phantomsection\label{refs}
\begin{CSLReferences}{1}{0}
\bibitem[\citeproctext]{ref-Angaroni2025}
Angaroni, Peruzzi, Fabrizio. 2025. {``Translating Microbial Kinetics
into Quantitative Responses and Testable Hypotheses Using Kinbiont.''}
\emph{Nature Communications} 16 (6440).
\url{https://www.nature.com/articles/s41467-025-61592-6\#citeas}.

\bibitem[\citeproctext]{ref-Bayer2023}
Bayer, Franziska, Torben Griebel, Luca Borger, and Tobias Bollenbach.
2023. {``Comparing Methods to Estimate Growth Parameters from Microbial
Growth Curves.''} \emph{Nature Communications} 14 (362): 1--13.
\url{https://doi.org/10.1038/s41467-023-43696-z}.

\bibitem[\citeproctext]{ref-Bretz2025}
Bretz, Frank, Björn Bornkamp, and Thomas Dumortier. 2025.
{``Dose-Response Characterization: A Key to Success in Drug
Development.''} \emph{Clinical Trials} 22 (4): 384--92.
\url{https://doi.org/10.1177/17407745251350289}.

\bibitem[\citeproctext]{ref-Burkner2017}
Bürkner, Paul-Christian. 2017. {``Brms: An r Package for Bayesian
Multilevel Models Using Stan.''} \emph{Journal of Statistical Software}
80 (1): 1--28.

\bibitem[\citeproctext]{ref-Degnan2025}
Degnan, Lisa M. AND Truong, David J. AND Bramer. 2025. {``Bmdrc: Python
Package for Quantifying Phenotypes from Chemical Exposures with
Benchmark Dose Modeling.''} \emph{PLOS Computational Biology} 21 (7).
\url{https://doi.org/10.1371/journal.pcbi.1013337}.

\bibitem[\citeproctext]{ref-Feldl2025}
Feldl, Medina, and Roberto Olayo Alarcon. 2025. \emph{DGrowthR:
Bacterial Growth Curve Analysis}.
\url{https://bio-datascience.github.io/DGrowthR/}.

\bibitem[\citeproctext]{ref-Gross2024}
Gross, Rotem, Muhittin Mungan, Suman G. Das, Melih Yüksel, Berenike
Maier, Tobias Bollenbach, Joachim Krug, and J. Arjan G. M. de Visser.
2024. {``Collective β-Lactam Resistance in Escherichia Coli Due to
β-Lactamase Release Upon Cell Death.''} \emph{bioRxiv}.
\url{https://doi.org/10.1101/2024.10.14.618215}.

\bibitem[\citeproctext]{ref-Kuosmanen2021}
Kuosmanen, Johannes AND Noble, Teemu AND Cairns. 2021. {``Drug-Induced
Resistance Evolution Necessitates Less Aggressive Treatment.''}
\emph{PLOS Computational Biology} 17 (9): 1--22.
\url{https://doi.org/10.1371/journal.pcbi.1009418}.

\bibitem[\citeproctext]{ref-Lukacisin2019}
Lukačišin, Martin, and Tobias Bollenbach. 2019. {``Emergent Gene
Expression Responses to Drug Combinations Predict Higher-Order Drug
Interactions.''} \emph{Cell Systems} 9 (5): 423--433.e3.
\url{https://doi.org/10.1016/j.cels.2019.10.004}.

\bibitem[\citeproctext]{ref-Lukacisinova2020}
Lukačišinová, Marta, Booshini Fernando, and Tobias Bollenbach. 2020.
{``Highly Parallel Lab Evolution Reveals That Epistasis Can Curb the
Evolution of Antibiotic Resistance.''} \emph{Nature Communications} 11:
3105. \url{https://doi.org/10.1038/s41467-020-16932-z}.

\bibitem[\citeproctext]{ref-Monod1949}
Monod, Jacques. 1949. {``The Growth of Bacterial Cultures.''}
\emph{Annual Review of Microbiology} 3 (1): 371--94.
\url{https://doi.org/10.1146/annurev.mi.03.100149.002103}.

\bibitem[\citeproctext]{ref-Petrungaro2025}
Petrungaro, Gabriela, Theresa Fink, Booshini Fernando, Gerrit Ansmann,
and Tobias Bollenbach. 2025. {``Function-Specific Epistasis Shapes
Evolutionary Trajectories Towards Antibiotic Resistance.''}
\emph{bioRxiv}. \url{https://doi.org/10.1101/2025.07.09.663857}.

\bibitem[\citeproctext]{ref-Pinheiro2023}
Pinheiro, José, and Douglas Bates. 2023. \emph{Mixed-Effects Models in s
and s-PLUS}. Springer.

\bibitem[\citeproctext]{ref-Ritz2015}
Ritz, Christian, Florent Baty, Jens C Streibig, and Daniel Gerhard.
2015. {``Dose-Response Analysis Using r.''} \emph{PLOS ONE} 10 (12):
e0146021.

\bibitem[\citeproctext]{ref-Wang2020}
Wang, Dennis, James Hensman, Ginte Kutkaite, Tzen S Toh, Ana Galhoz,
GDSC Screening Team, Jonathan R. Dry, et al. 2020. {``A Statistical
Framework for Assessing Pharmacological Responses and Biomarkers Using
Uncertainty Estimates.''} \emph{eLife} 9: e60352.
\url{https://doi.org/10.7554/eLife.60352}.

\end{CSLReferences}




\end{document}
